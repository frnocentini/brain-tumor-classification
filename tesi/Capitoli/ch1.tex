Al giorno d’oggi, con l’avvento delle tecnologie di \emph{imaging} biomedico,
 il numero di immagini che sono state catturate ed archiviate giorno dopo giorno negli ospedali 
 e nei laboratori sta crescendo sempre di più. Questo però deve avvenire di pari passo all'avanzamento tecnologico,
 di modo
  che questo fornisca sistemi  
  più robusti e all’avanguardia affinchè vengano raggiunti gli obiettivi di diagnosi e classificazione di 
  vari tipi di patologie. \\
Sulla scia di ciò, per assistere medici e specialisti, le immagini iniziano a essere usate 
ed utilizzate per allenare sistemi intelligenti. Pertanto in virtù di questa grande quantità di 
immagini mediche (ecografie, mammografie, RM...), l’uso di metodi basate sulle \emph{big data technologies}, 
come il \emph{Machine Learning} (ML) e l’\emph{Intelligenza Artificiale} è diventato fondamentale, 
anche e soprattutto come supporto all’equipe medica. 
Sono stati dunque proposti negli anni dei metodi per automatizzare il processo di analisi
 di immagine medica. 
Nel presente lavoro di tesi saranno analizzati ed utilizzati metodi di classificazione di 
immagini usando metodi di \emph{deep learning} (DL), in particolare si fa uso di \emph{reti neurali convoluzionali}.
 I metodi di DL non sono altro che un’evoluzione dei metodi di ML 
atti a trattare dati di più grande cardinalità e complessità come quelli inerenti al campo biomedico. \\
In particolare  in una prima parte verrà
 fatta una discriminazione tra soggetti affetti da polmonite e soggetti sani tramite l’utilizzo
  di radiografie al petto; in una seconda parte invece il sistema viene allenato in modo tale da 
  poter classificare immagini di risonanze magnetiche cerebrali tra 4 diverse diagnosi per 
  il soggetto nell’immagine: glioma, meningioma, tumore ipofisario e soggetto sano.
  Prima di tutto è necessario precedere la trattazione sulla realizzazione dei sistemi di classificazione con due sezioni
introduttive su ciò che rappresentano il ML e il DL e su quale è il funzionamento in linea teorica generale di una rete
neurale, per poter comprendere meglio quanto fatto nella fase sperimentale. \\
Più in dettaglio l'elaborato verrà suddiviso così:
\begin{itemize}
  \item \textbf{Capitolo 2:} si introducono i concetti di base sull'apprendimento automatico, 
  sul deep learning e le reti neurali e su ciò che significa addestrare una rete e farle acquisire capacità 
  di \emph{generalizzazione}. Si tratta di quelli che possono essere i principali problemi che si riscontrano
  nella realizzazione di sistemi di questo tipo e di cosa significa classificare.
  \item \textbf{Capitolo 3:} si analizza più nel dettaglio quella che è la teoria che definisce le strutture che sono state scelte
   nell'implementazione e che 
  hanno reso possibile 
  la classificazione d'immagine: le \textbf{CNN}.
  \item \textbf{Capitolo 4:} si introducono i linguaggi e le applicazioni utilizzate per elaborare i dati e
  per lavorare sull'apprendimento delle reti neurali, dunque si sottolinea quale è stato l'ambiente di lavoro.
  \item \textbf{Capitolo 5:} si descrivono le fasi per realizzazione del sistema di classificazione, a partire dal
 descrizione dei dataset fino ad arrivare alla visualizzazione delle previsioni sulle immagini stesse.
 \item \textbf{Capitolo 6:} si fa un breve resoconto dei risultati ottenuti e di un possibile sviluppo che può
 essere apportato sui sistemi realizzati. 
 \item \textbf{Capitolo 7:} si riportano due esempi di codice utilizzati per lavorare sulle immagini e classificarle.
\end{itemize}
