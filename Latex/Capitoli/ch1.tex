Al giorno d’oggi, con l’avvento delle tecnologie di imaging biomedico,
 il numero di immagini che sono state catturate ed archiviate giorno dopo giorno negli ospedali 
 e nei laboratori sta crescendo sempre di più. Con questo però, di pari passo all'avanzamento tecnologico, che preveda sistemi  
  più robusti e all’avanguardia affinchè vengano raggiunti gli obiettivi di diagnosi e classificazione di 
  vari tipi di patologie. 
Sulla scia di ciò, per assistere medici e specialisti, queste immagini possono essere usate 
ed utilizzate per allenare sistemi intelligenti. Pertanto in virtù di questa grande quantità di 
immagini mediche (ecografie, mammografie, MRI...), l’uso di metodi basate sulle \emph{big data technologies}, 
come il machine learning (ML) e l’Intelligenza Artificiale è diventato fondamentale, 
anche e soprattutto come supporto all’equipe medica. 
Sono stati dunque proposti negli anni dei metodi per automatizzare il processo di analisi
 di immagine medica. 
Nel presente lavoro di tesi saranno analizzati ed utilizzati metodi di classificazione di 
immagini usando metodi di deep learning. Questi ultimi non sono altro che un’evoluzione dei metodi di ML 
atti a trattare dati di grande cardinalità e complessità. \\
In particolare  in una prima parte verrà
 fatta una discriminazione tra soggetti affetti da polmonite e soggetti sani tramite l’utilizzo
  di radiografie al petto; in una seconda parte invece il sistema viene allenato in modo tale da 
  poter classificare immagini di risonanze magnetiche cerebrali tra 4 diverse diagnosi per 
  il soggetto nell’immagine: glioma, meningioma, tumore ipofisario e soggetto sano.

